%%%%%%%%%%%%%%%%%%%%%%%%%%%%%%%%%%%%%%%%%
% 模板資訊:
% 模板名稱:莊程翔 - CV
% 版本:1.0 (2023.03.21)
% 修改者:莊程翔 Ernie Cheng-Xiang Zhuang
% 編譯器:cwTeX (網路下載的僅有5套字體,需有吳聰敏教授提供之完整字體!)
%
% 原始模板的資訊:
% 模板名稱:Wilson Resume/CV
% 作者:1. Howard Wilson (https://github.com/watsonbox/cv_template_2004)
%      2. Vel (vel@latextemplates.com)
% 編譯器:XeLaTeX
% 授權:CC BY-NC-SA 3.0 (http://creativecommons.org/licenses/by-nc-sa/3.0/)
% 下載連結:http://www.latextemplates.com/template/wilson-resume-cv
%
% 製作本模板之目的:
% 為了讓 LaTeX 初學者能夠毫不費力地寫出整潔專業的 CV,因此我針對 Howard Wilson 和 Vel 製作的 CV 做了大幅度的修改及附上清楚明瞭的註解。
% 如果您有任何問題,可以透過以下兩種方式聯繫我: 
% 1. 網站:https://www.ernie-zhuang.com/contact
% 2. Email:erniezhuang1127@gmail.com
%%%%%%%%%%%%%%%%%%%%%%%%%%%%%%%%%%%%%%%%%

%----------------------------------------------------------------------------------------
%	封包與文檔配置
%----------------------------------------------------------------------------------------

% 設定為 A4 紙的大小及上下左右的邊界
\usepackage[a4paper, left=3.18cm, right=3.18cm, top=2.54cm, bottom=2.54cm]{geometry} 

% 超連結的封包
\usepackage[colorlinks, bookmarks = false]{hyperref}

%% 設定各種超連結的顏色
\hypersetup{
	linkcolor = red,
	citecolor = blue,
	filecolor = blue,
	urlcolor = blue
} 

% 禁止節(section)編號
\setcounter{secnumdepth}{0} 

% 輸出的字體是以 OT1 編碼
\usepackage[OT1]{fontenc} 

%% 設定中文字體
\usepackage{MinionPro}
\usepackage{MyriadPro}
\usepackage[scaled=0.85]{beramono} 

% 自訂字體顏色的封包
\usepackage{color} 

%% 自訂一個顏色
\definecolor{Myblue}{RGB}{57,108,144}

% 允許自行定義標題的封包
\usepackage{sectsty}

%% 設定節(section)為自訂的顏色
\sectionfont{\color{Myblue}} 

% 設定每段不向內縮進
\setlength\parindent{0pt} 

% 設定不向內縮進的標籤
\newenvironment{itemize-noindent}
{\setlength{\leftmargini}{0em}\begin{itemize}}
{\end{itemize}}

% 自訂方形的項目標籤
\newcommand{\sqbullet}{~\vrule height 1ex width .8ex depth -.2ex} 

% 註釋掉大部分的封包
\usepackage{comment}

%----------------------------------------------------------------------------------------
%	頁首
%----------------------------------------------------------------------------------------

\renewcommand{\title}[1]{
{\huge{\color{Myblue}\textbf{#1}}}\\ % Header section name and color
\rule{\textwidth}{0.5mm}\\ % Rule under the header
}

%----------------------------------------------------------------------------------------
%	研究志趣
%-----------------------------------------------------------------------------------------

\newcommand{\interests}[1]{
\begin{tabbing}
\hspace{5mm} \= \kill
#1
\end{tabbing}
\vspace{-10mm}
}

\newcommand{\interest}[1]{\sqbullet \> \textbf{#1}\\[3pt]} % Define the item name

%----------------------------------------------------------------------------------------
%	電腦技能
%----------------------------------------------------------------------------------------

\newcommand{\skills}[2]{
\begin{tabbing}
\hspace{5mm} \= \kill
\sqbullet \>\+ \textbf{#1} \\
\begin{minipage}{\textwidth}
\vspace{2mm}
#2
\end{minipage}
\end{tabbing}
}

%----------------------------------------------------------------------------------------
%	進行中的計畫
%----------------------------------------------------------------------------------------

\newcommand{\Progress}[1]{
\begin{tabbing}
\hspace{5mm} \= \kill
\sqbullet \>\+ 
\begin{minipage}[t]{14.14cm} % 21(width of A4) - 3.18(left & right margin)*2 - 0.5(item)
#1
\end{minipage}
\end{tabbing}
}


%----------------------------------------------------------------------------------------
%	標籤式區塊(Tabbed Block)
%----------------------------------------------------------------------------------------

\newcommand{\tabbedblock}[2]{
\begin{tabbing}
\hspace{3.64cm} \= \kill
{\bf #1}\> 
\begin{minipage}[t]{11cm} % 21(width of A4) - 3.18(left & right margin)*2 - 3.92(time box)
#2
\end{minipage}
\end{tabbing}
}

%----------------------------------------------------------------------------------------
%	更新資訊
%----------------------------------------------------------------------------------------

\newcommand{\update}[1]
{\raggedleft
\par \vfill \noindent {\small #1 (\today)}
}
